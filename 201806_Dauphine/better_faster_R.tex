\documentclass[10pt,ignorenonframetext,]{beamer}
\setbeamertemplate{caption}[numbered]
\setbeamertemplate{caption label separator}{: }
\setbeamercolor{caption name}{fg=normal text.fg}
\beamertemplatenavigationsymbolsempty
\usepackage{lmodern}
\usepackage{amssymb,amsmath}
\usepackage{ifxetex,ifluatex}
\usepackage{fixltx2e} % provides \textsubscript
\ifnum 0\ifxetex 1\fi\ifluatex 1\fi=0 % if pdftex
  \usepackage[T1]{fontenc}
  \usepackage[utf8]{inputenc}
\else % if luatex or xelatex
  \ifxetex
    \usepackage{mathspec}
  \else
    \usepackage{fontspec}
  \fi
  \defaultfontfeatures{Ligatures=TeX,Scale=MatchLowercase}
\fi
% use upquote if available, for straight quotes in verbatim environments
\IfFileExists{upquote.sty}{\usepackage{upquote}}{}
% use microtype if available
\IfFileExists{microtype.sty}{%
\usepackage{microtype}
\UseMicrotypeSet[protrusion]{basicmath} % disable protrusion for tt fonts
}{}
\newif\ifbibliography
\hypersetup{
            pdftitle={(A bit of) Advanced R},
            pdfauthor={Julien Chiquet},
            pdfborder={0 0 0},
            breaklinks=true}
\urlstyle{same}  % don't use monospace font for urls
\usepackage{color}
\usepackage{fancyvrb}
\newcommand{\VerbBar}{|}
\newcommand{\VERB}{\Verb[commandchars=\\\{\}]}
\DefineVerbatimEnvironment{Highlighting}{Verbatim}{commandchars=\\\{\}}
% Add ',fontsize=\small' for more characters per line
\usepackage{framed}
\definecolor{shadecolor}{RGB}{248,248,248}
\newenvironment{Shaded}{\begin{snugshade}}{\end{snugshade}}
\newcommand{\KeywordTok}[1]{\textcolor[rgb]{0.13,0.29,0.53}{\textbf{#1}}}
\newcommand{\DataTypeTok}[1]{\textcolor[rgb]{0.13,0.29,0.53}{#1}}
\newcommand{\DecValTok}[1]{\textcolor[rgb]{0.00,0.00,0.81}{#1}}
\newcommand{\BaseNTok}[1]{\textcolor[rgb]{0.00,0.00,0.81}{#1}}
\newcommand{\FloatTok}[1]{\textcolor[rgb]{0.00,0.00,0.81}{#1}}
\newcommand{\ConstantTok}[1]{\textcolor[rgb]{0.00,0.00,0.00}{#1}}
\newcommand{\CharTok}[1]{\textcolor[rgb]{0.31,0.60,0.02}{#1}}
\newcommand{\SpecialCharTok}[1]{\textcolor[rgb]{0.00,0.00,0.00}{#1}}
\newcommand{\StringTok}[1]{\textcolor[rgb]{0.31,0.60,0.02}{#1}}
\newcommand{\VerbatimStringTok}[1]{\textcolor[rgb]{0.31,0.60,0.02}{#1}}
\newcommand{\SpecialStringTok}[1]{\textcolor[rgb]{0.31,0.60,0.02}{#1}}
\newcommand{\ImportTok}[1]{#1}
\newcommand{\CommentTok}[1]{\textcolor[rgb]{0.56,0.35,0.01}{\textit{#1}}}
\newcommand{\DocumentationTok}[1]{\textcolor[rgb]{0.56,0.35,0.01}{\textbf{\textit{#1}}}}
\newcommand{\AnnotationTok}[1]{\textcolor[rgb]{0.56,0.35,0.01}{\textbf{\textit{#1}}}}
\newcommand{\CommentVarTok}[1]{\textcolor[rgb]{0.56,0.35,0.01}{\textbf{\textit{#1}}}}
\newcommand{\OtherTok}[1]{\textcolor[rgb]{0.56,0.35,0.01}{#1}}
\newcommand{\FunctionTok}[1]{\textcolor[rgb]{0.00,0.00,0.00}{#1}}
\newcommand{\VariableTok}[1]{\textcolor[rgb]{0.00,0.00,0.00}{#1}}
\newcommand{\ControlFlowTok}[1]{\textcolor[rgb]{0.13,0.29,0.53}{\textbf{#1}}}
\newcommand{\OperatorTok}[1]{\textcolor[rgb]{0.81,0.36,0.00}{\textbf{#1}}}
\newcommand{\BuiltInTok}[1]{#1}
\newcommand{\ExtensionTok}[1]{#1}
\newcommand{\PreprocessorTok}[1]{\textcolor[rgb]{0.56,0.35,0.01}{\textit{#1}}}
\newcommand{\AttributeTok}[1]{\textcolor[rgb]{0.77,0.63,0.00}{#1}}
\newcommand{\RegionMarkerTok}[1]{#1}
\newcommand{\InformationTok}[1]{\textcolor[rgb]{0.56,0.35,0.01}{\textbf{\textit{#1}}}}
\newcommand{\WarningTok}[1]{\textcolor[rgb]{0.56,0.35,0.01}{\textbf{\textit{#1}}}}
\newcommand{\AlertTok}[1]{\textcolor[rgb]{0.94,0.16,0.16}{#1}}
\newcommand{\ErrorTok}[1]{\textcolor[rgb]{0.64,0.00,0.00}{\textbf{#1}}}
\newcommand{\NormalTok}[1]{#1}

% Prevent slide breaks in the middle of a paragraph:
\widowpenalties 1 10000
\raggedbottom

\AtBeginPart{
  \let\insertpartnumber\relax
  \let\partname\relax
  \frame{\partpage}
}
\AtBeginSection{
  \ifbibliography
  \else
    \let\insertsectionnumber\relax
    \let\sectionname\relax
    \frame{\sectionpage}
  \fi
}
\AtBeginSubsection{
  \let\insertsubsectionnumber\relax
  \let\subsectionname\relax
  \frame{\subsectionpage}
}

\setlength{\parindent}{0pt}
\setlength{\parskip}{6pt plus 2pt minus 1pt}
\setlength{\emergencystretch}{3em}  % prevent overfull lines
\providecommand{\tightlist}{%
  \setlength{\itemsep}{0pt}\setlength{\parskip}{0pt}}
\setcounter{secnumdepth}{0}
\usepackage{../resources/themeBeamer}

\usepackage{etextools}
% make space after console-output smaller:
\setlength{\OuterFrameSep}{2pt}
\makeatletter
\preto{\@verbatim}{\topsep=-0pt \partopsep=0pt }
\makeatother

\graphicspath{{../resources/common_figs/}}
\usepackage{tikz}
\usetikzlibrary{calc,shapes,backgrounds,arrows,automata,shadows,positioning}
\tikzstyle{every state}=[fill=red,draw=none,scale=0.7,font=\small,text=white]
\tikzstyle{every edge}=[-,shorten >=1pt,auto,thin,draw]
\tikzstyle{alertstate}=[fill=bleu]
\definecolor{genecolor}{RGB}{94,135,173}

\let\oldtitle\title
\subtitle{\huge\oldtitle\normalsize}
\title{\currentCourse}
\institute{\currentInstitute}

\date{\currentDate}

\AtBeginSection{
  \begin{frame}<beamer>
    \frametitle{Outline}
    \framesubtitle{\insertpart}
    \tableofcontents[currentsection,currentsubsection, subsectionstyle=show/shaded/hide]  
  \end{frame}
}

\AtBeginSubsection{
  \begin{frame}<beamer>
    \frametitle{Outline}
    \framesubtitle{\insertpart}
    \tableofcontents[currentsection,currentsubsection, subsectionstyle=show/shaded/hide]  
  \end{frame}
}

\AtBeginSubsubsection{
  \begin{frame}<beamer>
    \frametitle{Outline}
    \framesubtitle{\insertpart}
    \tableofcontents[currentsection,currentsubsection, subsectionstyle=show/shaded/hide]  
  \end{frame}
}

\let\oldtitlepage\titlepage
\renewcommand{\titlepage}{%
    \oldtitlepage
    \vfill

\begin{tikzpicture}[remember picture,overlay]
  \node [xshift=2cm,yshift=1.5cm] at (current page.south west)    {\includegraphics[width=2.5cm]{logo_dauphine}};
  \node [yshift=-3.5cm] at (current page.center)    {\includegraphics[width=2cm]{sticker_sotr}};
  \node [xshift=-2cm,yshift=1.35cm] at (current page.south east)   {\includegraphics[width=2.5cm]{logo_inra}};
\end{tikzpicture}
}

\title{(A bit of) Advanced R}
\subtitle{Part 1 - towards better \texttt{R}-base programming}
\author{Julien Chiquet}
\institute{\url{http://github/jchiquet/CourseAdvancedR}}
\date{Université Paris Dauphine, Juin 2018}

\begin{document}
\frame{\titlepage}

\begin{frame}
  \frametitle{Outline}
  \tableofcontents[currentsection, sectionstyle=show/show,subsectionstyle=hide]
\end{frame}

\begin{frame}{References}

\begin{itemize}
\tightlist
\item
  R Core Team (2017): A Language and Environment for Statistical
  Computing \url{https://www.R-project.org/}
\item
  Wickham (2014): Advanced R, retrieved from
  \url{http://adv-r.had.co.nz/}
\item
  Gillespie \& Lovelace (2016): efficient R programming
  \url{https://bookdown.org/csgillespie/efficientR/}
\end{itemize}

\end{frame}

\begin{frame}{Prerequisites}

\begin{block}{Data Structure in base R}

\begin{enumerate}
\def\labelenumi{\arabic{enumi}.}
\tightlist
\item
  Atomic vector (integer, double, logical, character)
\item
  Recursive vector (list)
\item
  Factors
\item
  Matrices and array
\item
  Data Frame
\end{enumerate}

\rsa Creation, Basic Operation, Manipulation, Representation

\end{block}

\begin{block}{Resources}

\begin{itemize}
\tightlist
\item
  Advanced R, chapters I.2, I.3 (Wickham, 2014,
  \url{http://adv-r.had.co.nz/})
\item
  An introduction to R programmming
  \url{http://julien.cremeriefamily.info/teachings_L3BI_ISV51.html}
\end{itemize}

\end{block}

\end{frame}

\section{Function, Functionals}\label{function-functionals}

\begin{frame}[fragile,allowframebreaks]{The \texttt{{[}a-z{]}*pply}
family}

\begin{block}{Example with factors (\texttt{tapply})}

\scriptsize

\begin{Shaded}
\begin{Highlighting}[]
\NormalTok{data <-}\StringTok{ }\KeywordTok{rnorm}\NormalTok{(}\DecValTok{100}\NormalTok{)}
\NormalTok{sexe <-}\StringTok{ }\KeywordTok{factor}\NormalTok{( }\KeywordTok{sample}\NormalTok{(}\KeywordTok{c}\NormalTok{(}\StringTok{"H"}\NormalTok{,}\StringTok{"F"}\NormalTok{), }\DecValTok{100}\NormalTok{, }\DataTypeTok{replace =} \OtherTok{TRUE}\NormalTok{))}
\NormalTok{mean.}\DecValTok{1}\NormalTok{ <-}\StringTok{ }\KeywordTok{tapply}\NormalTok{(data, sexe, mean) ## good}
\NormalTok{mean.}\DecValTok{2}\NormalTok{ <-}\StringTok{ }\KeywordTok{c}\NormalTok{()  ## complicated}
\ControlFlowTok{for}\NormalTok{ (l }\ControlFlowTok{in} \KeywordTok{levels}\NormalTok{(sexe))}
\NormalTok{  mean.}\DecValTok{2}\NormalTok{ <-}\StringTok{ }\KeywordTok{c}\NormalTok{(mean.}\DecValTok{2}\NormalTok{, }\KeywordTok{mean}\NormalTok{(data[sexe }\OperatorTok{==}\StringTok{ }\NormalTok{l]))}
\end{Highlighting}
\end{Shaded}

\normalsize

\end{block}

\begin{block}{Example with list or data.frame
(\texttt{sapply}/\texttt{lapply})}

\scriptsize

\begin{Shaded}
\begin{Highlighting}[]
\KeywordTok{data}\NormalTok{(oats, }\DataTypeTok{package =} \StringTok{"MASS"}\NormalTok{)}
\NormalTok{oats[}\DecValTok{1}\OperatorTok{:}\DecValTok{2}\NormalTok{, ]}
\end{Highlighting}
\end{Shaded}

\color{gray}

\begin{verbatim}##   B       V      N   Y
## 1 I Victory 0.0cwt 111
## 2 I Victory 0.2cwt 130
\end{verbatim}

\begin{Shaded}
\begin{Highlighting}[]
\KeywordTok{sapply}\NormalTok{(oats, is.factor) ## readable}
\end{Highlighting}
\end{Shaded}

\color{gray}

\begin{verbatim}##     B     V     N     Y 
##  TRUE  TRUE  TRUE FALSE
\end{verbatim}

\begin{Shaded}
\begin{Highlighting}[]
\ControlFlowTok{for}\NormalTok{ (c }\ControlFlowTok{in} \DecValTok{1}\OperatorTok{:}\KeywordTok{ncol}\NormalTok{(oats)) ## less readable (I think)}
    \KeywordTok{print}\NormalTok{(}\KeywordTok{is.factor}\NormalTok{(oats[,c]))}
\end{Highlighting}
\end{Shaded}

\color{gray}

\begin{verbatim}## [1] TRUE
## [1] TRUE
## [1] TRUE
## [1] FALSE
\end{verbatim}

\normalsize

\end{block}

\end{frame}

\begin{frame}[fragile,allowframebreaks]{The \texttt{do.call} function}

\begin{quote}
\begin{quote}
\begin{quote}
constructs and executes a function call from a name or a function and a
list of arguments to be passed to it
\end{quote}
\end{quote}
\end{quote}

Suppose you have the outputs of 100 simulations at your disposable,
stored in a list like that

\scriptsize

\begin{Shaded}
\begin{Highlighting}[]
\NormalTok{res <-}\StringTok{ }\KeywordTok{replicate}\NormalTok{(}\DecValTok{100}\NormalTok{, }\KeywordTok{rbind}\NormalTok{(}\KeywordTok{data.frame}\NormalTok{(}\DataTypeTok{method=}\StringTok{"lasso"}\NormalTok{, }\DataTypeTok{mse=}\KeywordTok{runif}\NormalTok{(}\DecValTok{1}\NormalTok{,.}\DecValTok{75}\NormalTok{,}\DecValTok{1}\NormalTok{) , }\DataTypeTok{timing=}\KeywordTok{runif}\NormalTok{(}\DecValTok{1}\NormalTok{)),}
                            \KeywordTok{data.frame}\NormalTok{(}\DataTypeTok{method=}\StringTok{"ridge"}\NormalTok{, }\DataTypeTok{mse=}\KeywordTok{runif}\NormalTok{(}\DecValTok{1}\NormalTok{,.}\DecValTok{5}\NormalTok{,.}\DecValTok{75}\NormalTok{), }\DataTypeTok{timing=}\KeywordTok{runif}\NormalTok{(}\DecValTok{1}\NormalTok{)),}
                            \KeywordTok{data.frame}\NormalTok{(}\DataTypeTok{method=}\StringTok{"bayes"}\NormalTok{, }\DataTypeTok{mse=}\KeywordTok{runif}\NormalTok{(}\DecValTok{1}\NormalTok{,.}\DecValTok{85}\NormalTok{,}\DecValTok{1}\NormalTok{) , }\DataTypeTok{timing=}\KeywordTok{runif}\NormalTok{(}\DecValTok{1}\NormalTok{,}\DecValTok{100}\NormalTok{,}\DecValTok{200}\NormalTok{))}
\NormalTok{                              ), }\DataTypeTok{simplify=}\OtherTok{FALSE}\NormalTok{)}
\end{Highlighting}
\end{Shaded}

\normalsize

\scriptsize

\begin{Shaded}
\begin{Highlighting}[]
\KeywordTok{class}\NormalTok{(res)}
\end{Highlighting}
\end{Shaded}

\color{gray}

\begin{verbatim}## [1] "list"
\end{verbatim}

\begin{Shaded}
\begin{Highlighting}[]
\NormalTok{res[[}\DecValTok{1}\NormalTok{]]}
\end{Highlighting}
\end{Shaded}

\color{gray}

\begin{verbatim}##   method       mse      timing
## 1  lasso 0.8761956   0.1202532
## 2  ridge 0.6308892   0.1159765
## 3  bayes 0.8508328 123.9241986
\end{verbatim}

\begin{Shaded}
\begin{Highlighting}[]
\KeywordTok{length}\NormalTok{(res)}
\end{Highlighting}
\end{Shaded}

\color{gray}

\begin{verbatim}## [1] 100
\end{verbatim}

\normalsize

How would you store them in a single data frame? \scriptsize

\begin{Shaded}
\begin{Highlighting}[]
\NormalTok{all.res <-}\StringTok{ }\KeywordTok{do.call}\NormalTok{(rbind, res)}
\KeywordTok{dim}\NormalTok{(all.res)}
\end{Highlighting}
\end{Shaded}

\color{gray}

\begin{verbatim}## [1] 300   3
\end{verbatim}

\normalsize

\end{frame}

\begin{frame}[fragile]{The Reduce function}

\begin{quote}
`Reduce' uses a binary function to successively combine the elements of
a given vector
\end{quote}

\rsa can be use to post-process your list of simulations obtained via
\texttt{mclapply} just like \texttt{do.call}

Say more\ldots{} (map, Reduce)

\end{frame}

\begin{frame}[fragile]{A Reduce example: ``jacknifing'' a lasso solution
path}

A single Lasso fit of the diabete data set \scriptsize

\begin{Shaded}
\begin{Highlighting}[]
\KeywordTok{library}\NormalTok{(glmnet)}
\KeywordTok{library}\NormalTok{(lars) }\CommentTok{# the diabetes data set (part of the lars package)}
\KeywordTok{data}\NormalTok{(diabetes)}
\NormalTok{y <-}\StringTok{ }\NormalTok{diabetes}\OperatorTok{$}\NormalTok{y}
\NormalTok{x <-}\StringTok{ }\NormalTok{diabetes}\OperatorTok{$}\NormalTok{x}
\NormalTok{n <-}\StringTok{ }\KeywordTok{length}\NormalTok{(y)}
\NormalTok{lasso <-}\StringTok{ }\KeywordTok{glmnet}\NormalTok{(x,y)}
\KeywordTok{plot}\NormalTok{(lasso)}
\end{Highlighting}
\end{Shaded}

\begin{center}\includegraphics[width=.6 extwidth]{figures/Reduce_load-1} \end{center}

\normalsize

\end{frame}

\begin{frame}[fragile,allowframebreaks]{A Reduce example: ``jacknifing''
a lasso solution path II}

Compute the regularization paths for all subsets, removing one
individual at once

\scriptsize

\begin{Shaded}
\begin{Highlighting}[]
\NormalTok{paths <-}\StringTok{ }\NormalTok{parallel}\OperatorTok{::}\KeywordTok{mclapply}\NormalTok{(}\DecValTok{1}\OperatorTok{:}\NormalTok{n, }\ControlFlowTok{function}\NormalTok{(i) \{}
    \KeywordTok{glmnet}\NormalTok{(x[}\OperatorTok{-}\NormalTok{i, ], y[}\OperatorTok{-}\NormalTok{i], }\DataTypeTok{lambda =}\NormalTok{ lasso}\OperatorTok{$}\NormalTok{lambda)}\OperatorTok{$}\NormalTok{beta}
\NormalTok{\}, }\DataTypeTok{mc.cores =} \DecValTok{4}\NormalTok{)}
\end{Highlighting}
\end{Shaded}

\normalsize

Computing the envelop around the average regularization path with Reduce

\scriptsize

\begin{Shaded}
\begin{Highlighting}[]
\NormalTok{mean.path <-}\StringTok{ }\KeywordTok{Reduce}\NormalTok{(}\StringTok{"+"}\NormalTok{, paths)}\OperatorTok{/}\NormalTok{n}
\NormalTok{sdev.path <-}\StringTok{ }\KeywordTok{sqrt}\NormalTok{(}\KeywordTok{Reduce}\NormalTok{(}\StringTok{"+"}\NormalTok{, }\KeywordTok{lapply}\NormalTok{(paths, }\ControlFlowTok{function}\NormalTok{(path) path}\OperatorTok{**}\DecValTok{2}\NormalTok{))}\OperatorTok{/}\NormalTok{n }\OperatorTok{-}\StringTok{ }
\StringTok{                  }\NormalTok{mean.path}\OperatorTok{**}\DecValTok{2}\NormalTok{)}
\end{Highlighting}
\end{Shaded}

\normalsize

\scriptsize

\begin{Shaded}
\begin{Highlighting}[]
\KeywordTok{library}\NormalTok{(ggplot2)}
\KeywordTok{library}\NormalTok{(reshape2)}
\NormalTok{mean.path <-}\StringTok{ }\KeywordTok{t}\NormalTok{(}\KeywordTok{as.matrix}\NormalTok{(mean.path)); }\KeywordTok{rownames}\NormalTok{(mean.path) <-}\StringTok{ }\NormalTok{lasso}\OperatorTok{$}\NormalTok{lambda}
\NormalTok{sdev.path <-}\StringTok{ }\KeywordTok{t}\NormalTok{(}\KeywordTok{as.matrix}\NormalTok{(sdev.path)); }\KeywordTok{rownames}\NormalTok{(sdev.path) <-}\StringTok{ }\NormalTok{lasso}\OperatorTok{$}\NormalTok{lambda}
\NormalTok{dplot <-}\StringTok{ }\KeywordTok{cbind}\NormalTok{(reshape2}\OperatorTok{::}\KeywordTok{melt}\NormalTok{(mean.path), reshape2}\OperatorTok{::}\KeywordTok{melt}\NormalTok{(sdev.path)[, }\DecValTok{3}\NormalTok{])}
\KeywordTok{colnames}\NormalTok{(dplot) <-}\StringTok{ }\KeywordTok{c}\NormalTok{(}\StringTok{"lambda"}\NormalTok{,}\StringTok{"predictor"}\NormalTok{,}\StringTok{"mean"}\NormalTok{,}\StringTok{"sdev"}\NormalTok{)}
\NormalTok{ggobj <-}\StringTok{ }\KeywordTok{ggplot}\NormalTok{(dplot, }\KeywordTok{aes}\NormalTok{(}\DataTypeTok{x=}\NormalTok{lambda, }\DataTypeTok{y=}\NormalTok{mean, }\DataTypeTok{group=}\NormalTok{predictor, }\DataTypeTok{color=}\NormalTok{predictor)) }\OperatorTok{+}\StringTok{ }\KeywordTok{geom_smooth}\NormalTok{(}\KeywordTok{aes}\NormalTok{(}\DataTypeTok{ymin=}\NormalTok{mean}\OperatorTok{-}\NormalTok{sdev,}\DataTypeTok{ymax=}\NormalTok{mean}\OperatorTok{+}\NormalTok{sdev))}
\KeywordTok{print}\NormalTok{(ggobj }\OperatorTok{+}\StringTok{ }\KeywordTok{coord_trans}\NormalTok{(}\DataTypeTok{x=}\StringTok{"log10"}\NormalTok{))}
\end{Highlighting}
\end{Shaded}

\begin{center}\includegraphics[width=.8\textwidth]{figures/Reduce_plot-1} \end{center}

\normalsize

\end{frame}

\section{\texorpdfstring{Good and bad practices in
\texttt{R}}{Good and bad practices in R}}\label{good-and-bad-practices-in-r}

\begin{frame}[fragile,allowframebreaks]{Vectorize any algebraic
operation}

\emphase{Example:} compute \(\exp(x) = \sum_{k=0}^{n} \frac{x^k}{k!}\)

\scriptsize

\begin{Shaded}
\begin{Highlighting}[]
\NormalTok{## the good way}
\NormalTok{exp_vec <-}\StringTok{ }\ControlFlowTok{function}\NormalTok{(x, n)\{}
\NormalTok{  res <-}\StringTok{ }\KeywordTok{sum}\NormalTok{(x}\OperatorTok{^}\NormalTok{(}\DecValTok{0}\OperatorTok{:}\NormalTok{n)}\OperatorTok{/}\KeywordTok{c}\NormalTok{(}\DecValTok{1}\NormalTok{,}\KeywordTok{cumprod}\NormalTok{(}\DecValTok{1}\OperatorTok{:}\NormalTok{n)))}
\NormalTok{  res}
\NormalTok{\}}
\NormalTok{## the sad/bad/less readable way}
\NormalTok{exp_loop <-}\StringTok{ }\ControlFlowTok{function}\NormalTok{(x, n)\{}
\NormalTok{  res <-}\StringTok{ }\DecValTok{1}
  \ControlFlowTok{for}\NormalTok{ (k }\ControlFlowTok{in} \DecValTok{1}\OperatorTok{:}\NormalTok{n) res <-}\StringTok{ }\NormalTok{res }\OperatorTok{+}\StringTok{ }\DecValTok{2}\OperatorTok{^}\NormalTok{k}\OperatorTok{/}\KeywordTok{factorial}\NormalTok{(k)}
\NormalTok{  res}
\NormalTok{\}}
\end{Highlighting}
\end{Shaded}

\normalsize

\scriptsize

\begin{Shaded}
\begin{Highlighting}[]
\KeywordTok{autoplot}\NormalTok{(}\KeywordTok{microbenchmark}\NormalTok{(}\DataTypeTok{vec =} \KeywordTok{exp_vec}\NormalTok{(}\DecValTok{2}\NormalTok{, }\DecValTok{100}\NormalTok{), }\DataTypeTok{loop =} \KeywordTok{exp_loop}\NormalTok{(}\DecValTok{2}\NormalTok{, }\DecValTok{100}\NormalTok{)))}
\end{Highlighting}
\end{Shaded}

\begin{center}\includegraphics[width=.8\textwidth]{figures/vec_benchmark-1} \end{center}

\normalsize

\end{frame}

\begin{frame}[fragile,allowframebreaks]{Vectorize, even for
non-algebraic operation}

\scriptsize

\begin{Shaded}
\begin{Highlighting}[]
\NormalTok{month_year_apply <-}\StringTok{ }\ControlFlowTok{function}\NormalTok{(year) \{}
  \KeywordTok{sapply}\NormalTok{(month.name, }\ControlFlowTok{function}\NormalTok{(month) }\KeywordTok{paste}\NormalTok{(month, year, }\DataTypeTok{sep =} \StringTok{"_"}\NormalTok{))  }
\NormalTok{\}}

\NormalTok{month_year_outer <-}\StringTok{ }\ControlFlowTok{function}\NormalTok{(year) \{}
  \KeywordTok{outer}\NormalTok{(month.name, year, }\DataTypeTok{FUN =}\NormalTok{ paste, }\DataTypeTok{sep =} \StringTok{'_'}\NormalTok{)}
\NormalTok{\}}
\KeywordTok{head}\NormalTok{(}\KeywordTok{month_year_outer}\NormalTok{(}\KeywordTok{c}\NormalTok{(}\DecValTok{2010}\NormalTok{, }\DecValTok{2013}\NormalTok{)), }\DecValTok{3}\NormalTok{)}
\end{Highlighting}
\end{Shaded}

\color{gray}

\begin{verbatim}##      [,1]            [,2]           
## [1,] "January_2010"  "January_2013" 
## [2,] "February_2010" "February_2013"
## [3,] "March_2010"    "March_2013"
\end{verbatim}

\normalsize

\scriptsize

\begin{Shaded}
\begin{Highlighting}[]
\KeywordTok{autoplot}\NormalTok{(}\KeywordTok{microbenchmark}\NormalTok{(}
  \DataTypeTok{sapply =} \KeywordTok{month_year_apply}\NormalTok{(}\KeywordTok{c}\NormalTok{(}\DecValTok{2011}\NormalTok{, }\DecValTok{2013}\NormalTok{)),}
  \DataTypeTok{outer  =} \KeywordTok{month_year_outer}\NormalTok{(}\KeywordTok{c}\NormalTok{(}\DecValTok{2011}\NormalTok{, }\DecValTok{2013}\NormalTok{)), }
  \DataTypeTok{times =} \DecValTok{100}\NormalTok{))}
\end{Highlighting}
\end{Shaded}

\begin{center}\includegraphics[width=.8\textwidth]{figures/vec_outer_plot-1} \end{center}

\normalsize

\end{frame}

\begin{frame}[fragile,allowframebreaks]{Compile your functions with
\texttt{base::compiler}}

If you cannot avoid a loop, you will save some time

\scriptsize

\begin{Shaded}
\begin{Highlighting}[]
\NormalTok{cumsum.R <-}\StringTok{ }\ControlFlowTok{function}\NormalTok{(n) \{}
\NormalTok{  x <-}\StringTok{ }\KeywordTok{rnorm}\NormalTok{(n)}
  \KeywordTok{cumsum}\NormalTok{(x)}
\NormalTok{\}}

\NormalTok{cumsum.me <-}\StringTok{ }\ControlFlowTok{function}\NormalTok{(n) \{}
\NormalTok{  x <-}\StringTok{ }\KeywordTok{rnorm}\NormalTok{(n)}
\NormalTok{  res <-}\StringTok{ }\DecValTok{0}
  \ControlFlowTok{for}\NormalTok{ (i }\ControlFlowTok{in} \DecValTok{1}\OperatorTok{:}\KeywordTok{length}\NormalTok{(x)) }
\NormalTok{    res <-}\StringTok{ }\NormalTok{res }\OperatorTok{+}\StringTok{ }\NormalTok{x[i]}
\NormalTok{  res}
\NormalTok{\}}

\NormalTok{cumsum.cmp <-}\StringTok{ }\NormalTok{compiler}\OperatorTok{::}\KeywordTok{cmpfun}\NormalTok{(cumsum.me)}

\KeywordTok{autoplot}\NormalTok{(}
  \KeywordTok{microbenchmark}\NormalTok{(}
    \KeywordTok{cumsum.R}\NormalTok{(}\DecValTok{1000}\NormalTok{),}
    \KeywordTok{cumsum.me}\NormalTok{(}\DecValTok{1000}\NormalTok{),}
    \KeywordTok{cumsum.cmp}\NormalTok{(}\DecValTok{1000}\NormalTok{), }
    \DataTypeTok{times=}\DecValTok{1000}\NormalTok{)}
\NormalTok{)}
\end{Highlighting}
\end{Shaded}

\begin{center}\includegraphics[width=.8\textwidth]{figures/unnamed-chunk-5-1} \end{center}

\normalsize

\rsa Can be set automatically with \texttt{compiler::enableJIT(3)}

\end{frame}

\begin{frame}[fragile]{Preallocate whenever it is possible}

\scriptsize

\begin{Shaded}
\begin{Highlighting}[]
\NormalTok{grow <-}\StringTok{ }\ControlFlowTok{function}\NormalTok{(n) \{vec <-}\StringTok{ }\KeywordTok{numeric}\NormalTok{(}\DecValTok{0}\NormalTok{); }\ControlFlowTok{for}\NormalTok{ (i }\ControlFlowTok{in} \DecValTok{1}\OperatorTok{:}\NormalTok{n) vec <-}\StringTok{ }\KeywordTok{c}\NormalTok{(vec,i)\}}
\NormalTok{loop <-}\StringTok{ }\ControlFlowTok{function}\NormalTok{(n) \{vec <-}\StringTok{ }\KeywordTok{numeric}\NormalTok{(n); }\ControlFlowTok{for}\NormalTok{ (i }\ControlFlowTok{in} \DecValTok{1}\OperatorTok{:}\NormalTok{n) vec[i] <-}\StringTok{ }\NormalTok{i\}}
\NormalTok{vect <-}\StringTok{ }\ControlFlowTok{function}\NormalTok{(n) \{}\DecValTok{1}\OperatorTok{:}\NormalTok{n\}}
\end{Highlighting}
\end{Shaded}

\normalsize

\scriptsize

\begin{center}\includegraphics[width=.8\textwidth]{figures/do.prealloc-1} \end{center}

\normalsize

\end{frame}

\begin{frame}[fragile,allowframebreaks]{Do not stack objects}

Even if it is tempting when the final size is unknown.

\scriptsize

\begin{Shaded}
\begin{Highlighting}[]
\NormalTok{simu.stack <-}\StringTok{ }\ControlFlowTok{function}\NormalTok{(x) \{ ## x is a n x p matrix}
\NormalTok{  out <-}\StringTok{ }\KeywordTok{data.frame}\NormalTok{(}\DataTypeTok{mean =} \KeywordTok{numeric}\NormalTok{(}\DecValTok{0}\NormalTok{), }\DataTypeTok{sd =} \KeywordTok{numeric}\NormalTok{(}\DecValTok{0}\NormalTok{))}
  \ControlFlowTok{for}\NormalTok{ (i }\ControlFlowTok{in} \DecValTok{1}\OperatorTok{:}\NormalTok{n) }
\NormalTok{    out <-}\StringTok{ }\KeywordTok{rbind}\NormalTok{(out, }\KeywordTok{data.frame}\NormalTok{(}\DataTypeTok{mean =} \KeywordTok{mean}\NormalTok{(x[i,]), }\DataTypeTok{sd =} \KeywordTok{sd}\NormalTok{(x[i, ])) )}
  \KeywordTok{return}\NormalTok{(out)}
\NormalTok{\}}

\NormalTok{simu.df <-}\StringTok{ }\ControlFlowTok{function}\NormalTok{(x) \{}
\NormalTok{  out <-}\StringTok{ }\KeywordTok{data.frame}\NormalTok{(}\DataTypeTok{mean =} \KeywordTok{numeric}\NormalTok{(n), }\DataTypeTok{sd =} \KeywordTok{numeric}\NormalTok{(n))}
  \ControlFlowTok{for}\NormalTok{ (i }\ControlFlowTok{in} \DecValTok{1}\OperatorTok{:}\NormalTok{n)}
\NormalTok{    out[i, ] <-}\StringTok{ }\KeywordTok{c}\NormalTok{(}\DataTypeTok{mean =} \KeywordTok{mean}\NormalTok{(x[i,]), }\DataTypeTok{sd =} \KeywordTok{sd}\NormalTok{(x[i, ])) }
  \KeywordTok{return}\NormalTok{(out)}
\NormalTok{\}}

\NormalTok{simu.list <-}\StringTok{ }\ControlFlowTok{function}\NormalTok{(x) \{}
\NormalTok{  my.list <-}\StringTok{ }\KeywordTok{lapply}\NormalTok{(}\DecValTok{1}\OperatorTok{:}\NormalTok{n, }\ControlFlowTok{function}\NormalTok{(i) }\KeywordTok{c}\NormalTok{(}\KeywordTok{mean}\NormalTok{(x[i,]), }\KeywordTok{sd}\NormalTok{(x[i, ])))}
\NormalTok{  out <-}\StringTok{ }\KeywordTok{data.frame}\NormalTok{(}\KeywordTok{do.call}\NormalTok{(rbind, my.list))}
  \KeywordTok{colnames}\NormalTok{(out) <-}\StringTok{ }\KeywordTok{c}\NormalTok{(}\StringTok{"mean"}\NormalTok{,}\StringTok{"sd"}\NormalTok{)  }
  \KeywordTok{return}\NormalTok{(out)}
\NormalTok{\}}
\end{Highlighting}
\end{Shaded}

\normalsize

\scriptsize

\begin{Shaded}
\begin{Highlighting}[]
\NormalTok{n <-}\StringTok{ }\DecValTok{1000}\NormalTok{; p <-}\StringTok{ }\DecValTok{10}\NormalTok{; x <-}\StringTok{ }\KeywordTok{matrix}\NormalTok{(}\KeywordTok{rnorm}\NormalTok{(n}\OperatorTok{*}\NormalTok{p), n, p)}
\KeywordTok{autoplot}\NormalTok{(}\KeywordTok{microbenchmark}\NormalTok{(}\KeywordTok{simu.stack}\NormalTok{(x), }\KeywordTok{simu.df}\NormalTok{(x), }\KeywordTok{simu.list}\NormalTok{(x), }\DataTypeTok{times=}\DecValTok{20}\NormalTok{))}
\end{Highlighting}
\end{Shaded}

\begin{center}\includegraphics[width=.8\textwidth]{figures/stack.roll-1} \end{center}

\normalsize

\end{frame}

\begin{frame}{References}

\hypertarget{refs}{}
\hypertarget{ref-Rmarkdown}{}
Allaire, J., Xie, Y., McPherson, J., Luraschi, J., Ushey, K., Atkins,
A., \ldots{} Chang, W. (2018). \emph{Rmarkdown: Dynamic documents for
r}. Retrieved from \url{https://CRAN.R-project.org/package=rmarkdown}

\hypertarget{ref-Rinferno}{}
Burns, P. (2012). \emph{The r inferno}. Lulu. com. Retrieved from
\url{http://www.burns-stat.com/documents/books/the-r-inferno/}

\hypertarget{ref-cookbookR}{}
Chang, W. (2012). \emph{R graphics cookbook: Practical recipes for
visualizing data}. `` O'Reilly Media, Inc.'' Retrieved from
\url{http://www.cookbook-r.com/Graphs/}

\hypertarget{ref-rcpp}{}
Eddelbuettel, D. (2013). \emph{Seamless r and c++ integration with
rcpp}. Springer. Retrieved from \url{http://dirk.eddelbuettel.com}

\hypertarget{ref-reproducibleR}{}
Gandrud, C. (2016). \emph{Reproducible research with r and r studio}.
Chapman; Hall/CRC. Retrieved from
\url{https://github.com/christophergandrud/Rep-Res-Book}

\hypertarget{ref-efficientR}{}
Gillespie, C., \& Lovelace, R. (2016). \emph{Efficient r programming}.
`` O'Reilly Media, Inc.'' Retrieved from
\url{https://bookdown.org/csgillespie/efficientR/}

\hypertarget{ref-Rbase}{}
R Core Team. (2017). \emph{R: A language and environment for statistical
computing}. Vienna, Austria: R Foundation for Statistical Computing.
Retrieved from \url{https://www.R-project.org/}

\hypertarget{ref-advr}{}
Wickham, H. (2014). \emph{Advanced r}. CRC Press. Retrieved from
\url{http://adv-r.had.co.nz/}

\hypertarget{ref-rpkg}{}
Wickham, H. (2015). \emph{R packages: Organize, test, document, and
share your code}. `` O'Reilly Media, Inc.'' Retrieved from
\url{http://r-pkgs.had.co.nz/}

\hypertarget{ref-ggplot2}{}
Wickham, H. (2016). \emph{Ggplot2: Elegant graphics for data analysis}.
Springer. Retrieved from \url{http://ggplot2.tidyverse.org/reference/}

\hypertarget{ref-r4ds}{}
Wickham, H., \& Grolemund, G. (2016). \emph{R for data science: Import,
tidy, transform, visualize, and model data}. `` O'Reilly Media, Inc.''
Retrieved from \url{http://r4ds.had.co.nz}

\hypertarget{ref-knitr}{}
Xie, Y. (2015). \emph{Dynamic documents with r and knitr} (Vol. 29). CRC
Press. Retrieved from \url{https://yihui.name/knitr/}

\end{frame}

\end{document}
